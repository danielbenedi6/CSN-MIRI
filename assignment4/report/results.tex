
The results of the quality of the fit of each of the non-linear models tested for each of the languages are contained on Tables \ref{tab:res_se}, \ref{tab:aic} and \ref{tab:aic_diff}. The quality of the fit of a model was measured with two types of metrics: $s$, residual standard error; and $AIC$ (Akaike information criterion). 

Both metrics are functions that quantify the discrepancy between the actual data and the model function. Consequently, the lower the value of these error functions, the more accurately a regression model fits. In particular, Table \ref{tab:res_se} contains the $s$, residual standard error values and table \ref{tab:aic} the AIC values obtained for each of the models, in each of the languages.

\begin{table}[!htb]
\centering
\resizebox{\columnwidth}{!}{
\begin{tabular}{llllllllllll}
Language & 0 & 1 & 2 & 3 & 4 & 5 & 1+ & 2+ & 3+ & 4+ & 5+ \\ \hline

 Arabic & 22.599 & 1.022 & 0.996 & 0.971 & 1.091 & 0.958 & 1.006 & 0.965 & 0.964 & 1.073 & 1.060 \\
 Basque & 6.380 & 0.477 & 0.476 & 0.535 & 0.524 & 0.481 & 0.477 & 0.482 & 0.931 & 0.519 & 0.488 \\
 Catalan & 16.624 & 0.424 & 0.417 & 0.445 & 0.482 & 0.404 & 0.419 & 0.411 & 0.475 & 0.472 & 0.521 \\
 Chinese & 6.219 & 1.050 & 1.023 & 0.961 & 1.119 & 0.974 & 1.036 & 0.990 & 0.964 & 1.106 & 1.220 \\
 Czech & 14.600 & 4.497 & 4.182 & 4.439 & 4.792 & 4.233 & 4.325 & 4.199 & 4.223 & 4.647 & 4.476 \\
 English & 14.642 & 0.828 & 0.832 & 0.953 & 0.951 & 0.834 & 0.833 & 0.833 & 1.427 & 0.912 & 0.838 \\
 Greek & 14.958 & 0.864 & 0.863 & 0.901 & 0.893 & 0.867 & 0.864 & 0.867 & 0.949 & 0.889 & 0.872 \\
 Hungarian & 10.376 & 0.833 & 0.836 & 1.093 & 1.301 & 0.841 & 0.836 & 0.841 & 1.317 & 1.045 & 0.846 \\
 Italian & 15.185 & 0.771 & 0.741 & 0.782 & 0.843 & 0.731 & 0.750 & 0.730 & 0.859 & 0.820 & 0.735 \\
 Turkish & 8.885 & 0.380 & 0.384 & 0.505 & 0.403 & 0.378 & 0.383 & 0.384 & 0.854 & 0.397 & 0.378 \\
\end{tabular}}
\caption{Residual standard error for each model \label{tab:res_se}}
\end{table}


\begin{table}[!htb]
\centering
\resizebox{\columnwidth}{!}{
\begin{tabular}{lllllll}
\multicolumn{1}{r}{\textbf{}} & \multicolumn{5}{c|}{\textbf{Model}}    &  \\ \cline{2-6}
\multicolumn{1}{c}{\textbf{Language}} &
  \multicolumn{1}{c}{D. Poisson} &
  \multicolumn{1}{c}{D. Geom} &
  \multicolumn{1}{c}{Zeta} &
  \multicolumn{1}{c}{Truncated Z} &
  \multicolumn{1}{l|}{Altmann} &
  $AIC_{best}$ \\ \hline
\multicolumn{1}{c}{Arabic} & 222316.204 & 24198.794 & 12.753 & 10.106 & \multicolumn{1}{l|}{0.000} & 61853.678 \\
\multicolumn{1}{c}{Basque} & 38422.596 & 8364.030 & 0.152 & 0.066 & \multicolumn{1}{l|}{0.000} & 27332.538 \\
\multicolumn{1}{c}{Catalan} & 885719.670 & 61915.772 & 47.994 & 34.746 & \multicolumn{1}{l|}{0.000} & 126776.094 \\
\multicolumn{1}{c}{Chinese} & 591632.099 & 48863.560 & 199.429 & 185.451 & \multicolumn{1}{l|}{0.000} & 132238.223 \\
\multicolumn{1}{c}{Czech} & 883983.366 & 91126.664 & 36.121 & 27.817 & \multicolumn{1}{l|}{0.000} & 204789.467 \\
\multicolumn{1}{c}{English} & 718960.209 & 45955.536 & 258.402 & 215.189 & \multicolumn{1}{l|}{0.000} & 120243.434 \\
\multicolumn{1}{c}{Greek} & 145963.408 & 17133.537 & 3.129 & 0.000 & \multicolumn{1}{l|}{4.797} & 36275.129 \\
\multicolumn{1}{c}{Hungarian} & 433637.814 & 53469.625 & 0.000 & 1.724 & \multicolumn{1}{l|}{1.972} & 81358.519 \\
\multicolumn{1}{c}{Italian} & 234004.631 & 22877.764 & 0.355 & 0.000 & \multicolumn{1}{l|}{1.940} & 39279.960 \\
\multicolumn{1}{c}{Turkish} & 171839.447 & 24615.351 & 0.000 & 1.977 & \multicolumn{1}{l|}{1.997} & 39935.050 \\
\end{tabular}}
\caption{AIC difference of the models per language} \label{tab:aic}
\end{table}

Table \ref{tab:aic_diff} contains for each of the languages the differences in $\Delta$ AIC of each of the models with respect to the model that gives the best AIC.
\begin{table}[!htb]
\centering
\resizebox{\columnwidth}{!}{
\begin{tabular}{llllllllllll}
Language & 0 & 1 & 2 & 3 & 4 & 5 & 1+ & 2+ & 3+ & 4+ & 5+ \\ \hline

 Arabic & 755.580 & 13.525 & 8.300 & 2.209 & 29.227 & 0.000 & 10.791 & 1.780 & 1.356 & 26.209 & 25.155 \\
 Basque & 216.864 & 0.000 & 0.873 & 10.596 & 7.914 & 2.667 & 1.013 & 2.702 & 58.113 & 8.112 & 4.667 \\
 Catalan & 710.880 & 7.560 & 5.044 & 17.755 & 32.092 & 0.000 & 6.110 & 3.333 & 31.090 & 29.056 & 49.920 \\
 Chinese & 154.874 & 6.407 & 5.242 & 0.000 & 11.778 & 1.995 & 6.264 & 3.416 & 1.176 & 11.803 & 21.866 \\
 Czech & 218.075 & 11.803 & 0.000 & 10.504 & 23.006 & 3.100 & 5.932 & 1.703 & 2.681 & 18.583 & 13.878 \\
 English & 504.524 & 0.000 & 1.809 & 25.660 & 24.306 & 3.105 & 1.931 & 2.934 & 97.676 & 17.965 & 4.921 \\
 Greek & 489.504 & 0.000 & 0.870 & 8.214 & 5.660 & 2.690 & 1.030 & 2.642 & 18.164 & 5.989 & 4.637 \\
 Hungarian & 407.701 & 0.000 & 1.663 & 45.127 & 72.353 & 3.601 & 1.640 & 3.640 & 76.205 & 37.765 & 5.581 \\
 Italian & 512.902 & 7.219 & 1.554 & 10.660 & 22.382 & 0.106 & 3.512 & 0.000 & 27.624 & 18.779 & 1.952 \\
 Turkish & 358.249 & 0.000 & 1.961 & 33.200 & 6.628 & 1.293 & 1.892 & 2.952 & 94.152 & 6.019 & 2.304 \\
\end{tabular}}
\caption{AIC differences \label{tab:aic_diff}}
\end{table}


Table \ref{tab:params} contains the values of the parameters that give the best fit for each model and for each of the languages. 
\begin{table}[!htb]
\centering
\resizebox{\columnwidth}{!}{
\begin{tabular}{llllllllllllllllllllllll}
         & \multicolumn{23}{l}{Model} \\ \cline{2-24} 
         & \multicolumn{1}{l|}{1} & \multicolumn{2}{l|}{2}     & \multicolumn{2}{l|}{3}     & \multicolumn{1}{l|}{4} & \multicolumn{1}{l|}{5}         & \multicolumn{2}{l|}{1+}    & \multicolumn{3}{l|}{2+}        & \multicolumn{3}{l|}{3+}        & \multicolumn{2}{l}{4+} & \multicolumn{4}{l}{5+} \\
Language & \multicolumn{1}{l|}{b} & a & \multicolumn{1}{l|}{b} & a & \multicolumn{1}{l|}{c} & \multicolumn{1}{l|}{a} & a & b & \multicolumn{1}{l|}{c} & b & \multicolumn{1}{l|}{d} & a & b & \multicolumn{1}{l|}{d} & a & c & \multicolumn{1}{l|}{d} & a & d \\ \hline

 Arabic & \multicolumn{1}{l|}{0.351} & 0.432 & \multicolumn{1}{l|}{0.487} & 1.873 & \multicolumn{1}{l|}{0.007} & \multicolumn{1}{l|}{0.816} & 1.097 & 0.174 & \multicolumn{1}{l|}{0.005} & 0.385 & \multicolumn{1}{l|}{-0.477} & 0.013 & 1.129 & \multicolumn{1}{l|}{1.596} & 8.336 & 0.003 & \multicolumn{1}{l|}{-6.764} & 1.055 & -0.977  & 1.481 & 0.177 & 0.002 & \multicolumn{1}{l|}{-0.767}\\
 Basque & \multicolumn{1}{l|}{0.437} & 0.623 & \multicolumn{1}{l|}{0.487} & 1.600 & \multicolumn{1}{l|}{0.022} & \multicolumn{1}{l|}{0.951} & 0.702 & 0.425 & \multicolumn{1}{l|}{0.003} & 0.455 & \multicolumn{1}{l|}{-0.154} & 0.409 & 0.575 & \multicolumn{1}{l|}{0.372} & 19.900 & 0.002 & \multicolumn{1}{l|}{-18.580} & 1.083 & -0.412  & 0.744 & 0.410 & 0.003 & \multicolumn{1}{l|}{-0.052}\\
 Catalan & \multicolumn{1}{l|}{0.357} & 0.630 & \multicolumn{1}{l|}{0.409} & 1.800 & \multicolumn{1}{l|}{0.009} & \multicolumn{1}{l|}{0.817} & 0.938 & 0.251 & \multicolumn{1}{l|}{0.004} & 0.373 & \multicolumn{1}{l|}{-0.193} & 0.164 & 0.651 & \multicolumn{1}{l|}{0.954} & 17.862 & 0.001 & \multicolumn{1}{l|}{-16.367} & 0.940 & -0.474  & 1.156 & 0.251 & 0.001 & \multicolumn{1}{l|}{-0.512}\\
 Chinese & \multicolumn{1}{l|}{0.459} & 0.363 & \multicolumn{1}{l|}{0.663} & 1.289 & \multicolumn{1}{l|}{0.030} & \multicolumn{1}{l|}{0.990} & 1.246 & 0.017 & \multicolumn{1}{l|}{0.030} & 0.509 & \multicolumn{1}{l|}{-0.490} & 0.002 & 2.031 & \multicolumn{1}{l|}{1.637} & 0.335 & 0.054 & \multicolumn{1}{l|}{1.299} & 1.285 & -0.920  & 0.319 & 0.640 & -0.004 & \multicolumn{1}{l|}{0.308}\\
 Czech & \multicolumn{1}{l|}{0.525} & 0.044 & \multicolumn{1}{l|}{1.170} & 2.495 & \multicolumn{1}{l|}{0.011} & \multicolumn{1}{l|}{1.298} & 0.000 & 2.659 & \multicolumn{1}{l|}{-0.014} & 0.629 & \multicolumn{1}{l|}{-2.611} & 0.026 & 1.274 & \multicolumn{1}{l|}{0.584} & 55.278 & 0.002 & \multicolumn{1}{l|}{-55.380} & 2.712 & -5.369  & 0.000 & 2.580 & -0.015 & \multicolumn{1}{l|}{1.533}\\
 English & \multicolumn{1}{l|}{0.457} & 0.680 & \multicolumn{1}{l|}{0.473} & 2.527 & \multicolumn{1}{l|}{0.009} & \multicolumn{1}{l|}{1.131} & 0.824 & 0.402 & \multicolumn{1}{l|}{0.002} & 0.460 & \multicolumn{1}{l|}{-0.051} & 0.322 & 0.609 & \multicolumn{1}{l|}{0.795} & 40.647 & 0.001 & \multicolumn{1}{l|}{-38.789} & 1.453 & -1.216  & 0.243 & 0.677 & -0.001 & \multicolumn{1}{l|}{0.950}\\
 Greek & \multicolumn{1}{l|}{0.364} & 0.619 & \multicolumn{1}{l|}{0.420} & 1.805 & \multicolumn{1}{l|}{0.010} & \multicolumn{1}{l|}{0.829} & 0.718 & 0.360 & \multicolumn{1}{l|}{0.001} & 0.383 & \multicolumn{1}{l|}{-0.216} & 0.322 & 0.536 & \multicolumn{1}{l|}{0.553} & 14.571 & 0.002 & \multicolumn{1}{l|}{-13.241} & 0.969 & -0.528  & 0.245 & 0.604 & -0.001 & \multicolumn{1}{l|}{0.676}\\
 Hungarian & \multicolumn{1}{l|}{0.594} & 0.615 & \multicolumn{1}{l|}{0.612} & 2.948 & \multicolumn{1}{l|}{0.015} & \multicolumn{1}{l|}{1.715} & 0.577 & 0.637 & \multicolumn{1}{l|}{-0.001} & 0.599 & \multicolumn{1}{l|}{-0.115} & 0.664 & 0.598 & \multicolumn{1}{l|}{-0.123} & 37.671 & 0.002 & \multicolumn{1}{l|}{-36.046} & 2.620 & -3.342  & 0.472 & 0.689 & -0.001 & \multicolumn{1}{l|}{0.203}\\
 Italian & \multicolumn{1}{l|}{0.373} & 0.454 & \multicolumn{1}{l|}{0.503} & 1.832 & \multicolumn{1}{l|}{0.010} & \multicolumn{1}{l|}{0.845} & 0.749 & 0.318 & \multicolumn{1}{l|}{0.004} & 0.409 & \multicolumn{1}{l|}{-0.449} & 0.079 & 0.837 & \multicolumn{1}{l|}{1.065} & 47.040 & 0.001 & \multicolumn{1}{l|}{-45.684} & 1.085 & -0.899  & 0.149 & 0.677 & 0.001 & \multicolumn{1}{l|}{0.893}\\
 Turkish & \multicolumn{1}{l|}{0.413} & 0.735 & \multicolumn{1}{l|}{0.419} & 1.810 & \multicolumn{1}{l|}{0.015} & \multicolumn{1}{l|}{0.924} & 0.539 & 0.568 & \multicolumn{1}{l|}{-0.006} & 0.417 & \multicolumn{1}{l|}{-0.037} & 1.613 & 0.286 & \multicolumn{1}{l|}{-1.189} & 11.701 & 0.003 & \multicolumn{1}{l|}{-10.423} & 1.027 & -0.345  & 0.121 & 1.006 & -0.015 & \multicolumn{1}{l|}{0.767}\\
\end{tabular}}
\caption{Parameters of each fitted model \label{tab:params}}
\end{table}


Finally, for each of the languages two kind of plots were elaborated, one with the data samples and the different model curves and another with the data samples and only the curve of the selected as best model according to the error metrics. All these plots can be found in the Appendices \ref{appendix:plots} and \ref{appendix:best-model-plots}, respectively.