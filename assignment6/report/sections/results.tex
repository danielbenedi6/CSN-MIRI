
\subsection{Scaling of vertex degree}

Tables \ref{tab:BA_evolution_AIC}, \ref{tab:RA_evolution_AIC} and \ref{tab:NG_evolution_AIC} show the raw AIC values for the different functions modeling the growth of vertex degree over time. A lower value is better. Since it can be hard to infer the best model given the raw data, Tables \ref{tab:BA_evolution_diff}, \ref{tab:RA_evolution_diff} and \ref{tab:NG_evolution_diff} show the values relative to that of the best model, which will be 0.

Note that further details on the functions defining each model are given in Sections \ref{sec:mod_degseq} and \ref{sec:mod_scaling}.

\begin{table}[H]
\centering
\begin{tabular}{ccccccc}
                                  & \multicolumn{6}{c}{Vertex arrival time}                                     \\
\multicolumn{1}{c|}{Growth Model} & $10^0$   & $10^1$     & $10^2$     & $10^3$     & $10^4$      & $10^5$      \\ \hline
\multicolumn{1}{c|}{0}            & 13466729 & 12388632.8 & 10292925   & 7905469.7  & 5717260.3   & 3094972.14  \\
\multicolumn{1}{c|}{1}            & 1344240  & 330512.7   & -572703.1  & -892840.9  & -1552647.95 & -1896011.78 \\
\multicolumn{1}{c|}{2}            & 1343844  & 253638.8   & -581026.6  & -1890439.4 & -2883683.82 & -3795156.32 \\
\multicolumn{1}{c|}{3}            & 12264735 & 11193616   & 9080055    & 6643933.5  & 4410896.82  & 931830.53   \\
\multicolumn{1}{c|}{4}            & 14025181 & 12949158   & 10849575.1 & 8422310.1  & 6263055.27  & 2856641.09  \\
\multicolumn{1}{c|}{0+}           & 11270382 & 10200296.9 & 8081035    & 5645414.5  & 3336546.62  & -30605.42   \\
\multicolumn{1}{c|}{1+}           & 1344242  & 118238.2   & -620373.2  & -2261837.1 & -2534522.94 & -3434708.24 \\
\multicolumn{1}{c|}{2+}           & 1340790  & -128503.3  & -748752.6  & -2281540.1 & -2927399.01 & -3961792.37 \\
\multicolumn{1}{c|}{3+}           & 11283488 & 10212518.4 & 8092345.7  & 5655615.9  & 3347444.22  & -14376.05   \\
\multicolumn{1}{c|}{4+}           & 12284239 & 7026026.6  & 5005728.1  & 2552066.2  & -15020.58   & -4541211.64
\end{tabular}
\caption{Raw values of AIC of the different fits for the growth of vertex degree and for vertices with different arrival times for the \textbf{Barabási-Albert} model.}
\label{tab:BA_evolution_AIC}
\end{table}


%%%%%%%%%%%%%%%%%%%%%%%%%%%%%%%%%%%%%%%%%5


\begin{table}[H]
\centering
\begin{tabular}{ccccccc}
                                  & \multicolumn{6}{c}{Vertex arrival time}                                                                                                                                     \\
\multicolumn{1}{c|}{Growth Model} & \multicolumn{1}{c}{$10^0$} & \multicolumn{1}{c}{$10^1$} & \multicolumn{1}{c}{$10^2$} & \multicolumn{1}{c}{$10^3$} & \multicolumn{1}{c}{$10^4$} & \multicolumn{1}{c}{$10^5$} \\ \hline
\multicolumn{1}{c|}{0}            & 8250251.9                  & 8107313.2                  & 7719411.6                  & 7065672.5                  & 5976657.7                  & 3558414.4                  \\
\multicolumn{1}{c|}{1}            & 7197981                    & 7029556.2                  & 6581725.4                  & 5797791.99                 & 4369493.7                  & 489488.6                   \\
\multicolumn{1}{c|}{2}            & 46022.3                    & 180638.5                   & 397381.9                   & 660619.34                  & 838752.6                   & -491206.3                  \\
\multicolumn{1}{c|}{3}            & 3682816.7                  & 3726539                    & 3711418.8                  & 3610671.16                 & 3304714.6                  & 1609452.5                  \\
\multicolumn{1}{c|}{4}            & 181267.3                   & 1391763.9                  & 2516319.7                  & 3322646.23                 & 3713818.2                  & 2893321.2                  \\
\multicolumn{1}{c|}{0+}           & 3600713.5                  & 3637001.1                  & 3604182.6                  & 3466481.52                 & 3066513.5                  & 1070307.4                  \\
\multicolumn{1}{c|}{1+}           & 2787542                    & 2815714.9                  & 2763706.2                  & 2579871.29                 & 2070775.1                  & -193309.4                  \\
\multicolumn{1}{c|}{2+}           & -449654.6                  & -458499.9                  & -537092.5                  & 13400.48                   & -126333.7                  & -1845014.7                 \\
\multicolumn{1}{c|}{3+}           & 3601329.9                  & 3637607.1                  & 3604838.6                  & -                          & 3067402.8                  & 1076270.7                  \\
\multicolumn{1}{c|}{4+}           & -4195831.6                 & -4171181.4                 & -3494283.4                 & -3292412.9                 & -3977152                   & -3851501.3                
\end{tabular}
\caption{Raw values of AIC of the different fits for the growth of vertex degree and for vertices with different arrival times for the \textbf{Growth + Random Attachment} model.}
\label{tab:RA_evolution_AIC}
\end{table}


%%%%%%%%%%%%%%%%%%%%%%%%%%%%%%%%%%%%%%%%%5


\begin{table}[H]
\centering
\begin{tabular}{cccc}
                                  & \multicolumn{3}{c}{Vertex arrival time} \\
\multicolumn{1}{c|}{Growth Model} & $10^0$      & $10^1$      & $10^2$      \\ \hline
\multicolumn{1}{c|}{0}            & 2665444     & 4562882     & 3078313     \\
\multicolumn{1}{c|}{1}            & 15919141    & 15917584    & 15920137    \\
\multicolumn{1}{c|}{2}            & 2383116     & 2994873     & 2359965     \\
\multicolumn{1}{c|}{3}            & 14887015    & 14886800    & 14888863    \\
\multicolumn{1}{c|}{4}            & 17505263    & 17506635    & 17506867    \\
\multicolumn{1}{c|}{0+}           & 2487147     & 2853452     & 2557081     \\
\multicolumn{1}{c|}{1+}           & 14532070    & 14531893    & 14533158    \\
\multicolumn{1}{c|}{2+}           & 2379677     & 2569593     & 2357503     \\
\multicolumn{1}{c|}{3+}           & 8059022     & 8072553     & 8073492     \\
\multicolumn{1}{c|}{4+}           & 16354605    & 16077101    & 16356210   
\end{tabular}
\caption{Raw values of AIC of the different fits for the growth of vertex degree and for vertices with different arrival times for the \textbf{No Growth + Preferential Attachment} model.}
\label{tab:NG_evolution_AIC}
\end{table}
\begin{table}[H]
\centering
\begin{tabular}{ccccccc}
                                  & \multicolumn{6}{c}{Vertex arrival time}                                       \\
\multicolumn{1}{c|}{Growth Model} & $10^0$       & $10^1$     & $10^2$     & $10^3$      & $10^4$     & $10^5$    \\ \hline
\multicolumn{1}{c|}{0}            & 12125939.427 & 12517136.1 & 11041677.6 & 10187009.83 & 8644659.31 & 7636183.8 \\
\multicolumn{1}{c|}{1}            & 3450.533     & 459016.1   & 176049.5   & 1388699.19  & 1374751.06 & 2645199.9 \\
\multicolumn{1}{c|}{2}            & 3054.056     & 382142.1   & 167725.9   & 391100.68   & 43715.19   & 746055.3  \\
\multicolumn{1}{c|}{3}            & 10923945.669 & 11322119.3 & 9828807.6  & 8925473.58  & 7338295.83 & 5473042.2 \\
\multicolumn{1}{c|}{4}            & 12684391.264 & 13077661.3 & 11598327.7 & 10703850.23 & 9190454.28 & 7397852.7 \\
\multicolumn{1}{c|}{0+}           & 9929591.812  & 10328800.3 & 8829787.6  & 7926954.64  & 6263945.63 & 4510606.2 \\
\multicolumn{1}{c|}{1+}           & 3452.368     & 246741.5   & 128379.4   & 19702.98    & 392876.07  & 1106503.4 \\
\multicolumn{1}{c|}{2+}           & 0            & 0          & 0          & 0           & 0          & 579419.3  \\
\multicolumn{1}{c|}{3+}           & 9942698.539  & 10341021.7 & 8841098.3  & 7937155.98  & 6274843.23 & 4526835.6 \\
\multicolumn{1}{c|}{4+}           & 10943448.824 & 7154529.9  & 5754480.6  & 4833606.26  & 2912378.43 & 0        
\end{tabular}
\caption{Difference of AIC from the best model, for the different fits for the growth of vertex degree and for vertices with different arrival times for the \textbf{Barabási-Albert} model.}
\label{tab:BA_evolution_diff}
\end{table}


%%%%%%%%%%%%%%%%%%%%%%%%%%%%%%%%%%%%%%%%%5


\begin{table}[H]
\centering
\begin{tabular}{ccccccc}
                                  & \multicolumn{6}{c}{Vertex arrival time}                       \\
\multicolumn{1}{c|}{Growth Model} & $10^0$   & $10^1$   & $10^2$   & $10^3$   & $10^4$  & $10^5$  \\ \hline
\multicolumn{1}{c|}{0}            & 12446084 & 12278495 & 11213695 & 10358085 & 9953810 & 7409916 \\
\multicolumn{1}{c|}{1}            & 11393813 & 11200738 & 10076009 & 9090205  & 8346646 & 4340990 \\
\multicolumn{1}{c|}{2}            & 4241854  & 4351820  & 3891665  & 3953032  & 4815905 & 3360295 \\
\multicolumn{1}{c|}{3}            & 7878648  & 7897720  & 7205702  & 6903084  & 7281867 & 5460954 \\
\multicolumn{1}{c|}{4}            & 4377099  & 5562945  & 6010603  & 6615059  & 7690970 & 6744822 \\
\multicolumn{1}{c|}{0+}           & 7796545  & 7808182  & 7098466  & 6758894  & 7043666 & 4921809 \\
\multicolumn{1}{c|}{1+}           & 6983374  & 6986896  & 6257990  & 5872284  & 6047927 & 3658192 \\
\multicolumn{1}{c|}{2+}           & 3746177  & 3712682  & 2957191  & 3305813  & 3850818 & 2006487 \\
\multicolumn{1}{c|}{3+}           & 7797162  & 7808789  & 7099122  &          & 7044555 & 4927772 \\
\multicolumn{1}{c|}{4+}           & 0        & 0        & 0        & 0        & 0       & 0      
\end{tabular}
\caption{Difference of AIC from the best model, for the different fits for the growth of vertex degree and for vertices with different arrival times for the \textbf{Growth + Random Attachment} model.}
\label{tab:RA_evolution_diff}
\end{table}


%%%%%%%%%%%%%%%%%%%%%%%%%%%%%%%%%%%%%%%%%5


\begin{table}[H]
\centering
\begin{tabular}{cccc}
                                  & \multicolumn{3}{c}{Vertex arrival time}  \\
\multicolumn{1}{c|}{Growth Model} & $10^0$       & $10^1$     & $10^2$       \\ \hline
\multicolumn{1}{c|}{0}            & 285767.322   & 1993288.8  & 720810.214   \\
\multicolumn{1}{c|}{1}            & 13539464.645 & 13347990.9 & 13562634.04  \\
\multicolumn{1}{c|}{2}            & 3438.945     & 425279.9   & 2462.349     \\
\multicolumn{1}{c|}{3}            & 12507338.737 & 12317206.2 & 12531360.486 \\
\multicolumn{1}{c|}{4}            & 15125586.033 & 14937041.8 & 15149363.626 \\
\multicolumn{1}{c|}{0+}           & 107470.591   & 283858.3   & 199577.718   \\
\multicolumn{1}{c|}{1+}           & 12152393.14  & 11962299.5 & 12175655.439 \\
\multicolumn{1}{c|}{2+}           & 0            & 0          & 0            \\
\multicolumn{1}{c|}{3+}           & 5679345.288  & 5502960    & 5715989.132  \\
\multicolumn{1}{c|}{4+}           & 13974927.943 & 13507507.3 & 13998706.756
\end{tabular}
\caption{Difference of AIC from the best model, for the different fits for the growth of vertex degree and for vertices with different arrival times for the \textbf{No Growth + Preferential Attachment} model.}
\label{tab:NG_evolution_diff}
\end{table}

\subsection{Degree distribution}

Table \ref{tab:degseq_AIC} presents the raw AIC values  of the different distributions for the three graph models under consideration. Likewise, Table \ref{tab:degseq_diff} presents the difference with respect to the best model. Note that in both tables, \textit{BA} refers to tbe \textit{Barabási-Albert} model, \textit{G+RA} to the \textit{Growth + Random Attachment} model, and \textit{NG + PA} to the \textit{No Growth + Preferential Attachment} model.

\begin{table}[H]
\centering
\begin{tabular}{cccc}
                                         & \multicolumn{3}{c}{Graph Model} \\
\multicolumn{1}{c|}{Distribution}        & BA        & G+RA     & NG+PA    \\ \hline
\multicolumn{1}{c|}{Displaced Poisson}   & 400226.7  & 3864871  & 15009.96 \\
\multicolumn{1}{c|}{Displaced Geometric} & 270338.7  & 4306365  &          \\
\multicolumn{1}{c|}{Fixed Zeta}          & 473815    & 8019202  & 38073.58 \\
\multicolumn{1}{c|}{Zeta}                & 328183.8  & 5586216  & 23845.98 \\
\multicolumn{1}{c|}{Truncated Zeta}      & 882413    & 6307312  & 17965.17 \\
\multicolumn{1}{c|}{Altmann}             & 270340.9  & 4306371  & 13913.82
\end{tabular}
\caption{Raw AIC values for the different fits of the degree distribution of the graphs constructed with the different models.}
\label{tab:degseq_AIC}
\end{table}


\begin{table}[H]
\centering
\begin{tabular}{cccc}
                                         & \multicolumn{3}{c}{Graph Model}    \\
\multicolumn{1}{c|}{Distribution}        & BA         & G+RA      & NG+PA     \\ \hline
\multicolumn{1}{c|}{Displaced Poisson}   & 129888.01  & 0         & 1096.136  \\
\multicolumn{1}{c|}{Displaced Geometric} & 0.02539979 & 441493.1  &           \\
\multicolumn{1}{c|}{Fixed Zeta}          & 203476.32  & 4154331   & 24159.753 \\
\multicolumn{1}{c|}{Zeta}                & 57845.08   & 1721344.5 & 9932.157  \\
\multicolumn{1}{c|}{Truncated Zeta}      & 612074.26  & 2442440.7 & 4051.349  \\
\multicolumn{1}{c|}{Altmann}             & 0          & 441499.7  & 0        
\end{tabular}
\caption{Difference of AIC from the best model, for the different fits of the degree distribution of the graphs constructed with the different models.}
\label{tab:degseq_diff}
\end{table}


\subsection{Conclusion}

The results show that Model 2+ is the most suited to represent the scaling of vertex degree with the \textit{Barabási-Albert} model and the \textit{No Growth + Preferential Attachment}, while Model 4+ is the best fit for the \textit{Growth + Random Attachment}.

Regarding the degree distribution, Altmann is the best for the first and third cases, while the second graph model has a degree distribution that could be best modeled with a Poisson function.

Note, however, that we think these values are not correct and are a result of a mistake in our code that we have not been able to fix. We give more details on this in the next section.