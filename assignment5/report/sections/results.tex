Tables \ref{tab:sig_karate}, \ref{tab:sig_barabasi_albert},  \ref{tab:sig_enron} and \ref{tab:sig_dblp} show the significance metrics computed for the different networks using \verb|evaluate_significance| provided by the \verb|clusterAnalytics| R package. Tables \ref{tab:best_canonical} and \ref{tab:best_no_canonical} highlight which clustering algorithm performs best according to the different metrics. As can be seen, the best algorithm is highly dependent on the network itself.

Surprisingly, the known canonical clustering does not seem to outperform the best clustering algorithm in terms of the metrics considered for Barabási-Albert blocks.

Table \ref{tab:jaccard} shows the global Jaccard indices for each of the networks and algorithms. In case a canonical clustering was known, it was used as the baseline. For the Enron network, we compared the algorithms to clustering produced by the Label Propagation algorithm. In the case of DBLP we used Louvain as a benchmark. Note that the clustering algorithms use randomness in their decisions, so two computations of the same algorithm produced two different clusterings, which in turn lead to a Jaccard index lower than 1 for Louvain.


\begin{table}[ht]
\centering
\begin{tabular}{l|ll|ll}
\toprule
\textbf{Metric} & \multicolumn{2}{l|}{\textbf{Karate} }    & \multicolumn{2}{l}{\textbf{Barabasi Albert Blocks} }      \\ 
\hline
   & \multicolumn{1}{l|}{best} & second best &  \multicolumn{1}{l|}{best} & second best\\ \hline
   
  \big\uparrow Internal Density  & \multicolumn{1}{l|}{Edge Betweenness} & Walktrap &   \multicolumn{1}{l|}{Walktrap} & Spin Glass \\ 
 \big\uparrow Edges Inside  & \multicolumn{1}{l|}{ground truth} & Label Propagation + FG &  \multicolumn{1}{l|}{Label Propagation} & Edge Betweenness \\ 
 \big\uparrow Average Degree  &\multicolumn{1}{l|}{ground truth} & Label Propagation + FG & \multicolumn{1}{l|}{Label Propagation} & Louvain + SG\\ 
\hline
 \big\downarrow Expansion  & \multicolumn{1}{l|}{ground truth} & Label Propagation + FG &\multicolumn{1}{l|}{Label Propagation} & Louvain + SG\\ 
 \big\downarrow Cut Ratio  &  \multicolumn{1}{l|}{ground truth} & Label Propagation + FG & \multicolumn{1}{l|}{Louvain + SG + LP} & \\ 
\hline
 \big\downarrow Conductance  &  \multicolumn{1}{l|}{ground truth} & Label Propagation + FG & \multicolumn{1}{l|}{Label Propagation} & Lovain \\ 
 \big\downarrow Normalized Cut & \multicolumn{1}{l|}{ground truth} & Label Propagation + FG& \multicolumn{1}{l|}{Label Propagation} & Lovain + SG\\ 
 \big\downarrow Maximum ODF & \multicolumn{1}{l|}{ground truth} & Label Propagation + FG& \multicolumn{1}{l|}{Label Propagation} & Spin Glass\\ 
 \big\downarrow Average ODF  & \multicolumn{1}{l|}{ground truth} & Label Propagation + FG & \multicolumn{1}{l|}{Label Propagation} & Spin-Glass \\ 
   \hline
\end{tabular}
\caption{Best values for communities with ground truth}
\label{tab:best_canonical}
\end{table}


\begin{table}[ht]
\centering
\begin{tabular}{lll}
    \toprule
    \textbf{Metric} & \textbf{Enron} & DBLP \\
    \midrule
    \big\uparrow Internal Density & Edge Betweenness & Label Propagation\\
    \big\uparrow Edges Inside & Label Propagation & Spin-Glass \\
    \big\uparrow Average Degree & Label Propagation& Louvain \\
    \midrule
    \big\downarrow Expansion & Label Propagation & Louvain \\
    \big\downarrow Cut Ratio & Label Propagation & Louvain \\
    \midrule
    \big\downarrow Conductance & Label Propagation & Louvain \\
    \big\downarrow Normalized Cut & Label Propagation & Louvain \\
    \big\downarrow Maximum ODF & Label Propagation & Louvain    \\
    \big\downarrow Average ODF & Label Propagation & Louvain \\
   \bottomrule
\end{tabular}
\caption{Best values for communities without ground truth}
\label{tab:best_no_canonical}
\end{table}

% latex table generated in R 4.3.2 by xtable 1.8-4 package
% Wed Nov 22 15:27:58 2023
\begin{table}[ht]
\centering
\begin{tabular}{lrrrrrr}
  \hline
 & Louvain & Label Propagation & Walktrap & Edge Betweenness & Fast Greedy & Spin-Glass \\
  \hline
karate & 0.7977941 & 0.7977941 & 0.5853758 & 0.4040033 & 0.7977941 & 0.6147876 \\
  Barabasi-Albert & 0.8470966 & 0.2700000 & 0.7433218 & 0.8605959 & 0.8637888 & 0.8798209 \\
  ENRON & 0.1534839 & 1.0000000 & 0.2247313 & 0.1095882 & 0.4229562 & 0.1800507 \\
  DBLP & 0.7845611 & 0.2427005 & 0.2691897 & 0.5206746 & 0.6423933 & 0.3293205 \\
   \hline
\end{tabular}
\caption{Global jaccard indices for each data set and clustering algorithm}
\label{tab:jaccard}
\end{table}

% latex table generated in R 4.3.2 by xtable 1.8-4 package
% Wed Nov 22 15:27:58 2023
\begin{table}[ht]
\centering
\begin{tabular}{lllll}
  \hline
Network & Vertices & Edges & Mean.Degree & Density \\ 
  \hline
Karate & 34 & 78 & 4.5882 & 0.139 \\ 
  Barabasi-Albert & 200 & 800 & 8 & 0.0402 \\ 
  ENRON & 182 & 2097 & 23.044 & 0.1273 \\ 
  DBLP & 2130 & 4587 & 4.307 & 0.002 \\ 
   \hline
\end{tabular}
\caption{Summary of the properties of the networks used.}
\label{tab:summary}
\end{table}
